%% Template for a preprint Letter or Article for submission
%% to the journal Nature.
%% Written by Peter Czoschke, 26 February 2004
%%


\documentclass[12pt,twocolumn]{article}
\usepackage{times}
\renewcommand{\baselinestretch}{1.5} 
\newenvironment{affiliations}{%
    \setcounter{enumi}{1}%
    \setlength{\parindent}{0in}%
    \slshape\sloppy%
    \begin{list}{\upshape$^{\arabic{enumi}}$}{%
        \usecounter{enumi}%
        \setlength{\leftmargin}{0in}%
        \setlength{\topsep}{0in}%
        \setlength{\labelsep}{0in}%
        \setlength{\labelwidth}{0in}%
        \setlength{\listparindent}{0in}%
        \setlength{\itemsep}{0ex}%
        \setlength{\parsep}{0in}%
        }
    }{\end{list}\par\vspace{12pt}}

%% Redefine the abstract environment to be the first bold paragraph
\renewenvironment{abstract}{%
    \setlength{\parindent}{0in}%
    \setlength{\parskip}{0in}%
    \bfseries%
    }{\par\vspace{-6pt}}





%\documentclass{nature}
%\documentclass{article}


%% make sure you have the nature.cls and naturemag.bst files where
%% LaTeX can find them

\bibliographystyle{naturemag}

\title{Det \"ar bara ett litet stick i fingret}

%% Notice placement of commas and superscripts and use of &
%% in the author list

\author{Geraldine Lang$^{1}$, Liv Lomax$^1$, Stefan Lang$^3$,Lennart
Johansson$^2$}


\begin{document}

\onecolumn

\maketitle

\begin{affiliations}
 \item ST-l\"akare i Allm\"anmedicin, N\"asets L\"akargrupp
 \item Specialist l\"akare i Allm\"anmedicin, N\"asets L\"akargrupp
 \item Statistiker vid Divisionen f\"or Molekyl\"ar Hematologi, Lund universitet
\end{affiliations}

\begin{abstract}
Abstract is missing up to now!
\end{abstract}


\twocolumn

\newpage

\section{Backgrund}


R\"adsla och sm\"arta i samband med provtagning p\r{a} barn \"ar vanligt
f\"orekommande, men s\"allan n\r{a}got vi l\"akare beh\"over konfronteras med
n\"ar vi skickar ut barnet till labbet. Ju mindre barnet \"ar desto sv\r{a}rare
har det att f\"orst\r{a} n\"odv\"andigheten av en ibland sm\"artsam procedur och
reagerar med r\"adsla och oro. Inte bara fysisk skada kan leda till sm\"arta
utan \"aven f\"orv\"antad sm\"arta kan leda till oro och \r{a}ngest, som i sin
tur f\"orst\"arker sm\"artupplevelsen \cite{Carverius2014}.
Man trodde l\"ange att sm\r{a} barn inte kunde k\"anna sm\"arta p\r{a} samma
s\"att som vuxna d\r{a} deras nervsystem var omoget \cite{Rey1993}. Under de
senaste 30
\r{a}ren har forskning kring barns sm\"arta dock visat att att \"aven sm\r{a}
barn har ett minne av sm\"arta \cite{Anand2007,Fitzgerald2001,Schechter2003}.
Minnen av sm\"artsamma upplevelser kan bidra till negativa upplevelser av
v\r{a}rden och d\"armed f\"orsv\r{a}ra barnets framtida kontakter med
sjukv\r{a}rden \cite{vBayer2004}. Dessa aspekter ger oss l\"akare ett stort
ansvar och \"ar n\r{a}got som vi beh\"over beakta i v\r{a}r kliniska vardag.

En betydande andel av bes\"ok p\r{a} VC utg\"ors av sm\r{a} barn med
infektionssymtom. De patientn\"ara analyserna f\"or identifiering av
Betahemolytiska streptokocker grupp A (Strep-A) och C-reaktivt protein (CRP)
\"ar diagnostiska verktyg i l\"akarens kliniska vardag, som i b\"asta fall kan
minska l\"akarens os\"akerhet och f\"orb\"attra diagnoss\"attning vid
infektioner i \"oppenv\r{a}rden. Men ut\"over att vara f\"orknippat med sm\"arta
och oro \"ar tolkning av testresultaten inte helt okomplicerat.
Exempelvis ger analys av CRP, ett akutfasprotein som stiger vid
infektionssjukdomar och inflammation, s\"allan v\"agledning f\"or om antibiotika
beh\"ovs eller ej. Vid okomplicerad infektion utan allm\"anp\r{a}verkan har
testet ringa v\"arde och det ska helst ha g\r{a}tt 24 timmar efter insjuknandet
f\"or att testet skall vara tillf\"orlitligt. Vissa lokala allvarliga
infektioner ger heller inte kraftig stegring av CRP tidigt i f\"orloppet
\cite{Tecken2014}.

\"Aven identifiering av streptokocktonsilliter med snabbtestet Strep-A, medf\"or
diagnostiska dilemman. Enligt tillverkarna har testet en sensitivitet p\r{a} >
93\% och en specificitet p\r{a} $>$ 94\%. Ett problem \"ar dock att en betydande
andel av alla barn \"ar b\"arare av streptokocker vintertid, utan att vara
sjuka. D\r{a} merparten av halsinfektioner \"ar orsakade av virus riskerar
testet, om taget p\r{a} fel grund, att felklassificera virusinfektioner som
streptokocktonsilliter hos b\"arare.

Trots att det finns p\r{a}visat ett samband mellan h\"og f\"orskrivning av
antibiotika och h\"og anv\"andning av CRP och Strep-A \cite{Studie2014} har
snabbtesterna fortfarande stor anv\"andning i prim\"arv\r{a}rden. I en tidigare
studie fr\r{a}n Canada har man belyst barns upplevelser och erfarenheter av
ven\"os provtagning \cite{Fradet1990}. I studien har man bland annat visat ett
samband mellan barns \r{a}lder och upplevelser vid provtagning samt visat att
f\"or\"aldrar v\"al kan f\"oruts\"aga barnets v\"antade reaktion under
provtagning \cite{Fradet1990}. Samma studie har \"aven indikerat att
f\"or\"aldrars oro korrelerar med barnets stressupplevelse vid ven\"os
provtagning.

Vi k\"anner inte till n\r{a}gon studie som tittat p\r{a} barn och f\"or\"aldrars
erfarenheter av provtagning med de patientn\"ara analyserna CRP och Strep-A. Vi
k\"anner inte heller till att liknande studier genomf\"orts i
prim\"arv\r{a}rden.

Syftet med denna studie var att belysa barns sm\"artupplevelser vid provtagning.
Vi ville \"aven studera eventuellt samband mellan barns sm\"artupplevelser och
faktorer som \r{a}lder, k\"on, tidigare exponering f\"or provtagning och
f\"or\"aldrarnas f\"oruts\"agelse av barnets sm\"artreaktion.

\section{Vetenskapliga Fr\r{a}gest\"allningar}

Hur vanligt f\"orekommande \"ar sm\"arta vid patientn\"ara provtagning p\r{a}
N\"asets L\"akargrupp (NLG)?
Finns det skillnad i utfall (sj\"alvskattad sm\"arta) beroende p\r{a} barnets
\r{a}lder, k\"on, tidigare exponering f\"or provtagning? V\r{a}r hypotes var att
barnets sm\"artupplevelse avtar med stigande \r{a}lder.
Finns samband mellan utfall och f\"or\"aldrarnas f\"oruts\"agelse av barnets
reaktion, ev stickr\"adsla hos f\"or\"aldrarna? V\r{a}r hypotes var att
f\"or\"aldrar kan f\"orutse sitt barns sm\"artrespons.

\section{Method}

Barn i \r{a}ldersgruppen 0-17 \r{a}r, som skickade till labbet p\r{a} NLG f\"or
patientn\"ara analys (CRP, Strep-A), ingick i studien. \"Aven f\"or\"aldrar till
barnen som deltog ingick i studien. Deltagandet i studien var oberoende av
symtom barnet s\"okte f\"or och studien inkluderade \"aven barn med
underliggande kronisk sjukdom.

Innan provtagning, ute i v\"antrummet, presenterades syftet med studien f\"or
f\"or\"aldern och barnet med hj\"alp av ett informationsblad. F\"or\"aldern fick
fylla i en kort enk\"at kring vad barnet hade f\"or besv\"ar vid bes\"oket (t ex
feber, halsont, hosta) samt ange huruvida de upplevde att barnets
allm\"antillst\r{a}nd hade f\"ors\"amrats eller ej senaste dygnet. F\"or\"aldern
fick \"aven kryssa f\"or om man hade \"onskem\r{a}l om att det skulle tas prov i
samband med bes\"oket, om detta var n\r{a}got man f\"orv\"antade sig. I
enk\"aten ombads f\"or\"aldern uppge om barnet tagit n\r{a}gon prov p\r{a} VC
(stick i fingret, svalgprov) tidigare och om barnet blev ledsen i samband med
f\"orra provtagningen. F\"or\"aldern ombads \"aven skatta, enligt en 10 cm
visuell analog skala, hur dom trodde att barnet skulle reagera vid dagens
provtagning.
Ena \"anden av skalan definierades som “lugn, ingen oro� andra \"anden som
“mycket ledsen, mycket orolig�. Enligt denna skala ombads f\"or\"aldern \"aven
skatta vad dom sj\"alva tycker om att ta prov.

Direkt efter provtagning ombads f\"or\"aldern, enligt samma skala som ovan,
skatta hur deras barn tedde sig i v\"antrummet inf\"or provtagning samt barnets
reaktion under sj\"alva provtagningen.

Direkt efter provtagningen bad en sk\"oterska p\r{a} labbet \"aven barnet skatta
sin upplevda sm\"arta i samband med provtagningen. Barn har i forskning uppgett
att dom f\"oredrar sm\"artskalor med ansikten (facial pain scales, FPS)
framf\"or andra typer av sj\"alvskattningsinstrument \cite{vBayer2006}. Av denna
anledning valde vi att anv\"anda den v\"alk\"anda Wong-Baker FACES Sm\"artskala
i v\r{a}r studie (bild 1). Denna skala visar en serie ansikten d\"ar \"anden vid
0 visar ett glatt ansikte “ingen sm\"arta� och \"anden vid 10 visar ett
gr\r{a}tande ansikte “v\"arsta t\"ankbara sm\"arta�.


Sm\"arta \"ar alltid subjektiv och det \"ar en utmaning att utv\"ardera sm\"arta
hos barn.
Inom h\"also- och sjukv\r{a}rden \"ar sj\"alvskattning ett f\"orstahandsval vid
bed\"omning av sm\"arta. Ansiktsskalor har visat validitet fr\r{a}n 4 \r{a}rs
\r{a}lder (11, 12). D\r{a} barn i \r{a}ldersgruppen 0-3 \r{a}r inte \"ar kapabla
till sj\"alvskattning, har vi inte haft m\"ojlighet att analysera dessa
uppgifter enligt v\r{a}rt studieuppl\"agg. Barn yngre \"an 4 \r{a}r exkluderades
d\"arf\"or fr\r{a}n den delen av studien d\"ar barnet ombads skatta sin sm\"arta
i samband med provtagning. Av egen erfarenhet, som b\r{a}de l\"akare och
f\"or\"alder, tror vi dock att barn i \r{a}ldersgruppen 0-3 \r{a}r \"ar extra
vulnerabla vid provtagning. Vi bad d\"arf\"or f\"or\"aldrar fylla i enk\"aten
\"aven f\"or barn under 4 \r{a}r.

Vi valde att inte s\"atta n\r{a}gon \"ovre \r{a}ldersgr\"ans f\"or deltagande av
barn, f\"orutsatt att en f\"or\"alder medf\"oljde vid l\"akarbes\"oket och
provtagningen.

M\r{a}ls\"attningen var att samla in enk\"ater under \r{a}tminstone en s\"asong
med h\"og f\"orekomsts av infektioner hos barn. Vi samlade in totalt 156
enk\"ater.
Majoriteten av alla enk\"ater samlades in under perioden november-16 till
april-17.

Enk\"aterna utformats s\r{a} att det framgick vilket test, CRP eller Strep-A,
som analyserades. N\r{a}gra g\r{a}nger (n=6) hade l\"akaren ordinerat b\r{a}de
CRP och Strep-A. D\r{a} fick barnet skatta sin sm\"artresponsen enligt FRS f\"or
b\r{a}da proverna separat.

Resultat fr\r{a}n studien… kommer att beskrivas deskriptivt. F\"or att studera
samband mellan variabler kommer vi att anv\"anda analytisk statistik,
prelimin\"art Chi-tv\r{a}-test som \"ar en metod som har utvecklats f\"or att
analysera data d\r{a} variabler har ordinalskala.

Projektet var ett internt forskningsarbete enligt vetenskapliga principer.
Ans\"okan till etikn\"amnd genomf\"ordes d\"arf\"or ej. Verksamhetschefen gav
sin till\r{a}telse till att v\r{a}r enk\"atbaserade unders\"okning
genomf\"ordes. Insamlade uppgifter var helt anonyma och det \"ar inte m\"ojligt
att identifiera patienterna eller deras f\"or\"aldrar genom enk\"aten.
Deltagandet var helt frivilligt. Studien kommer inte att publiceras offentligt.

\section{Resultat}

\subsection{Studiepopulationen}
Totalt deltog 156 barn och deras f\"or\"aldrar/n\"arst\r{a}ende i studien. Av
156 barn var 87 flickor och 55 pojkar, medan k\"onet ej hade angivits f\"or 13
deltagare. De yngsta barnen utgjorde den st\"orsta andelen av v\r{a}r
studiepopulation. 92 barn (59\%) var $<$ 7 \r{a}r gamla. F\"or detaljerad
k\"ons- och \r{a}ldersf\"ordelning se \% tabell 1 och 2.

I 103 enk\"ater angavs mamma som medf\"oljande, i 39 enk\"ater angavs pappa. I 4
enk\"ater angavs att b\r{a}de mamma och pappa var med, medan annan
n\"arst\r{a}ende fyllde i enk\"aten i 5 fall. I 4 enk\"ater  saknades
information om medf\"oljande.

133 av 156 barn (85\%) hade tagit prov (CRP och/eller Strep-A) tidigare. Redan i
\r{a}ldersgruppen 0-3 \r{a}r hade X\% redan genomg\r{a}tt minst en provtagning
tidigare, se tabell …

Symtom fr\r{a}n luftv\"agarna (hosta, halsont, f\"orkylning, \"oronv\"ark
och/eller feber i kombination med n\r{a}got av dessa symtom) utgjorde den
vanligaste bes\"oksorsaken i v\r{a}r studiepopulation. \"ovriga bes\"oksorsaker
var oklar/isolerad feber, GI-symtom (kr\"akningar, diarré, magont utan
f\"orkylningssymtom) och UVI symtom. Tabell 3 visar f\"ordelning av
bes\"oksorsaker i populationen.

85\% av alla f\"or\"aldrar uppgav att dom hade f\"orv\"antat sig att det skulle
tas \% prov i samband med l\"akarbes\"oket. 57\% hade \"onskem\r{a}l om detta,
medan 36\% inte \% \"onskade att det skulle tas n\r{a}got prov (7\% svarade ej
p\r{a} fr\r{a}gan).

P\r{a} fr\r{a}gan r\"orande barnets allm\"antillst\r{a}nd svarade 34\% att
barnets allm\"antillst\r{a}nd hade f\"ors\"amrats senaste dygnet. 50\% uppgav
ett of\"or\"andrat allm\"antillst\r{a}nd medan 16\% svarade att barnets
allm\"antillst\r{a}nd hade f\"orb\"attrats.

Korrelation mellan variabler V\r{a}r hypotes var att barnets sm\"artupplevelse minskar
med stigande \r{a}lder. V\r{a}r studie visade att \r{a}lder \"ar en faktor som korrelerar v\"al
med hur barnet sj\"alv skattar sin sm\"arta vid provtagning f\"or CRP och Strep-A
(diagram xa). F\"or\"aldrars observerade ocks\r{a} mindre oro och obehag (diagram xb)
med stigande \r{a}lder. Deras observation av barnets reaktion under provtagning
korrelerade v\"al med barnets sj\"alvskattade sm\"arta (P $<$ 0,001).
F\"or\"aldrarna kunde endast kunde fyllde i fr\r{a}ga 9 i enk\"aten en g\r{a}ng, trots att barnet tog b\r{a}de CRP
och Strep-A, men resultatet var i princip of\"or\"andrat n\"ar denna lilla grupp barn
(n=6) exkulderades. I den l\"agsta \r{a}ldersgrupperna skattade barn i genomsnitt
m\r{a}ttlig sm\"arta (definierat som 4-7 p\r{a} FPS) vid provtagning f\"or CRP, medan \"ovriga
\r{a}ldersgrupper endast skattade lindrig sm\"arta (definierat som 0-3 p\r{a} FPS) vid
provtagning. Vid provtagning Strep-A l\r{a}g medelv\"ardet f\r{a}r skattad sm\"arta enligt
FPS p\r{a} $<$ 4 i samtliga \r{a}ldersgrupper. Spridningen i sj\"alvskattad
sm\"arta var, som framg\r{a}r i diagram x-x stor i flera \r{a}ldersgrupper.

\section{Diskussion}

V\r{a}r studie har visat att en stor andelen av barn som skickas f\"or provtagning p\r{a}
NLG \"ar yngre \"an 7 \r{a}r. Det \"ar oklart om \r{a}ldersf\"ordelningen i studiepopulationen
speglar \r{a}ldersf\"ordelningen bland barn som s\"oker med infektionssymtom p\r{a} NLG d\r{a}
endast barn som skulle skickas till labbet och deras f\"or\"aldrar ingick i studien.
En t\"ankbar m\"ojlighet till \r{a}ldersf\"ordelningen i studiepopulationen \"ar att l\"akare
\"ar mer ben\"agna att ta prov p\r{a} de yngsta barnen. Eventuella f\"orklaringar till
detta vore intressanta att belysa i en framtida studie. Likas\r{a} g\"aller detta
k\"onsf\"ordelningen i studiepopulationen. Det f\"orv\r{a}nade oss d\r{a} det generellt var
m\r{a}nga fler flickor \"an pojkar i v\r{a}r studiepopulation. \r{a}terigen, \"ar detta
representativt f\"or bes\"okspopulationen eller \"ar l\"akare mer ben\"agna att ta prov p\r{a}
flickor? Genom att identifiera de undergrupper av barn som p\r{a}verkas mest
negativt vid provtagning finns m\"ojlighet att rikta s\"arskild h\"ansyn till dessa
grupper, s\r{a}som extra f\"orberedelse inf\"or provtagning och eventuell intervention
under sj\"alva provtagningen (t ex distraktion).

V\r{a}r huvudhypotes var att \r{a}ldern korrelerar med barns sm\"artupplevelse och att sm\r{a}
barn utg\"or en undergrupp som beh\"over ..  Resultat fr\r{a}n studien visar att barn i
de yngre \r{a}ldersgrupperna \"ar mer vulnerabla vid provtagning baserat p\r{a} deras egen
sm\"artskattning. F\"or\"aldrarnas observation var ocks\r{a} att de yngsta barnen tedde
sig mer ledsna och oroliga i samband med provtagning.

Detta \"ar n\r{a}got som vi l\"akare b\"or beakta n\"ar vi ordinerar prov. I v\r{a}r studie hade
60.. \% av f\"or\"aldrarna uppgett att barnets AT var of\"or\"andrat eller f\"orb\"attrat.
D\r{a} m\r{a}ste man fr\r{a}ga sig om provtagning, som \"ar n\r{a}got de minsta barnen
f\"orknippar med…, verkligen \"ar n\"odv\"andig.

I studien framgick ocks\r{a} att f\"or\"aldrar \"ar \"ar bra p\r{a} att f\"orutse hur deras barn
kommer att reagera i samband med provtagning. Genom att efterh\"ora hur
f\"or\"aldrarna tror att deras barn kommer att  reagera finns en m\"ojlighet att f\r{a}nga
upp de barn som beh\"over extra st\"od och f\"orberedelse. Studien p\r{a}visade \"aven ett
samband mellan barnets oro i v\"antrummet inf\"or provtagningen och sm\"arta och
obehag vid provtagning. Med detta i \r{a}tanke tror vi att distraktion/intervention
redan i v\"antrummet, speciellt riktat mot \r{a}ldersgrupp 0-9 \r{a}r, kan minska
stressniv\r{a}n inf\"or provtagning och d\"armed barnets obehag vid sj\"alva
provtagningen.

\section{Konklusion}

Vi tror att en \"okad kunskap kring snabbtesterna kan bidra till f\"arre
slentrianm\"assigt best\"allda prover och d\"arigenom ett minskat on\"odigt lidande f\"or
barnet. Genom att identifiera de eventuella undergrupper av barn som p\r{a}verkas
mest negativt vid provtagning finns ocks\r{a} m\"ojligheter rikta s\"arskild h\"ansyn till
dessa grupper, s\r{a}som extra f\"orberedelse inf\"or provtagning och eventuell
intervention under sj\"alva provtagningen (t ex distraktion).


\begin{figure}
\caption{Each figure legend should begin with a brief title for
the whole figure and continue with a short description of each
panel and the symbols used. For contributions with methods
sections, legends should not contain any details of methods, or
exceed 100 words (fewer than 500 words in total for the whole
paper). In contributions without methods sections, legends should
be fewer than 300 words (800 words or fewer in total for the whole
paper).}
\end{figure}

<<<<<<< Upstream, based on branch 'master' of https://github.com/Geraldin77/Document.git
=======
\section*{Methods}
>>>>>>> 9dffad0 Small test - working?


%% Put the bibliography here, most people will use BiBTeX in
%% which case the environment below should be replaced with
%% the \bibliography{} command.
\newpage
\onecolumn

\begin{thebibliography}{1}
\bibitem{Carverius2014} Carverius U, Ljungman G. Sm\"artbehandling vid procedurer
hos barn- kort historik och dagsl\"aget. L\"akemedelsverket, bakgrundsdokumentation 2014;3:24-26. 
\bibitem{Rey1993} Rey R. The history of pain. 1ed. Cambridge, Massachusetts:
Harvard University Press 1993.
\bibitem{Anand2007}Anand K, Stevens B, McGarth PJ. Pain in Neonates and infants:
Pain Research and Clinical Management Series. Elseiver 2007.
\bibitem{Fitzgerald2001}Fitzgerald M, Beggs S. The neurobiology of pain:
developmental aspects. Neuroscientists         2001;7:246-57.
\bibitem{Schechter2003} Schechter N, Berde S, Yaster M. Pain in Infants,
Children, and Adolescents. Lippincott Williams \& Wilkins 2003.
\bibitem{vBayer2004} von Bayer CL, et al. Children memory for pain: overview and implications for practice. J.Pain 2004.
\bibitem{Tecken2014} Tecken p\r{a} allvarlig infektion hos barn. Ett kunskapsunderlag med f\"orslag till handl\"aggning i prim\"arv\r{a}rd. www.folkhalsomyndigheten.se 2014.
\bibitem{Studie2014} Studie \"over faktorer som p\r{a}verkar l\"akarens beteende vid f\"orskrivning av antibiotika. Resultat fr\r{a}n tv\r{a} beteendevetenskapliga studier. www.folkh\"alsomyndigheten.se 2014.
\bibitem{Fradet1990} Fradet C, McGarth PJ, Kay J, Adams S, Luke B. A prospective survey of reactions to blood        tests by children and adolescents. Pain. Elseiver 1990;40:53-60.
\bibitem{vBayer2006} Von Baeyer C.    Children´s self reports on pain intensity: Scale selection, limitations and            interpretation. Pain Res Management 2006;11:157-162.
\bibitem{Tsze2013} Tsze DS, von Bayer CL, Bullock B, et al. Validation of self-report pain scales in children.     Paediatrics 2013;132:971-9.
\bibitem{WHO2012} WHO. Persisting pain in children Package: Who Guidelines on Pharmacological Treatment   of Persisting Pain in Children with Medical Illnesses. 2012;ISBN 9241548126;30-31.

\end{thebibliography}


%% Here is the endmatter stuff: Supplementary Info, etc.
%% Use \item's to separate, default label is "Acknowledgements"


\begin{addendum}
 \item Put acknowledgements here.
 \item[Competing Interests] The authors declare that they have no
competing financial interests.
 \item[Correspondence] Correspondence and requests for materials
should be addressed to A.B.C.~(email: myaddress@nowhere.edu).
\end{addendum}

%%
%% TABLES
%%
%% If there are any tables, put them here.
%%

\end{document}
